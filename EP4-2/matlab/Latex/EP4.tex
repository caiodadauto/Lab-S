%%%%%%%%%%%%%%%%%%%%%%%%%%%%%%%%%%%%%%%%%%%%%%%%%%%%%
%     Nome: Caio Vinícius Dadauto                   %
%     Número USP(código): 7994808                   %
%     Curso: Laboratório de simulação e computação  %
%     Turma: Noturno                                %
%%%%%%%%%%%%%%%%%%%%%%%%%%%%%%%%%%%%%%%%%%%%%%%%%%%%%

\documentclass [a4paper,10pt]{article}
\newcommand{\n}[1]{\textbf{#1}}
\newcommand{\e}[1]{\textcolor{red}{#1}}
\linespread {1.5}
\usepackage[brazilian]{babel}
\usepackage[utf8]{inputenc}
\usepackage[T1]{fontenc}
\usepackage{booktabs}
\usepackage{amsfonts}
\usepackage{amsmath}
\usepackage{amssymb}
\usepackage{caption}
\usepackage{subcaption}
\usepackage{multirow}
\usepackage{fancyhdr}
\usepackage{verbatim}
\usepackage{alltt}
\usepackage{graphicx,xcolor}
\usepackage{listings}
\lstset{numbers=left,
    stepnumber=1,
    firstnumber=1,
    numberstyle=\tiny,
    extendedchars=false,
    breaklines=flase,
    tabsize=2,
    showtabs=true,
    tab=\textcolor{gray}{$\cdots$},
    keywordstyle=\color{blue},
    frame=tb,
    basicstyle=\footnotesize,
    stringstyle=\ttfamily,
showstringspaces=false}
\renewcommand{\lstlistingname}{Programa}
\renewcommand{\lstlistlistingname}{Lista de Listagens}
\usepackage[pdftex]{hyperref}
\hypersetup{colorlinks,%
linkcolor=red}

\begin{document}
\thispagestyle{fancy}
\fancyhf{}
\renewcommand{\footrulewidth}{0.0pt}
\renewcommand{\headrulewidth}{0.0pt}
\rhead{\bfseries {\scriptsize 7/06/2015}}
\cfoot{\bfseries \thepage}

\begin{flushleft}
    \begin{tabular}{ l l }
        \multirow{2}{*}{\rule{0.15\textwidth}{48pt}} & \hspace{-3.5mm}{\large Exercício de Programa 4:}\\[2mm] 
                                                     & \hspace{-3.5mm}{\huge \n{Full Bayesian Significance Test}}\\[-2.85mm]
                                                     & \hspace{-4.5mm}\rule{1.1\textwidth}{1.6pt}
    \end{tabular}
\end{flushleft}
\begin{center}
    \vspace{-2.5mm}
    \hspace{10mm}{\small\emph{Instituto de Matemática e Estatística da Universidade de São Paulo}}\\[0.5cm]
    \hspace{-5.5cm}\begin{tabular}{ l l }
        \n{Por} & \\[-2mm]
                & \hspace{-10mm}{\small Caio Vinícius Dadauto$\qquad$7994808}\\
        \n{Professor} & \\[-2mm]
                      & \hspace{-10mm}{\small Julio Michael Stern}\\[4mm]
    \end{tabular}
\end{center}
\vspace{2cm}

\section{O modelo}
Seja o seguinte problema:
\begin{eqnarray*}
    \Theta & = & \{(\alpha, \beta, \gamma)\;\in\;]0, \infty[\;\mathrm{x}\;[1, \infty[\;\mathrm{x}\;[0, \infty[\}\\
    \Theta_0 & = & \{(\alpha, \beta, \gamma)\;\in\;\Theta\;|\;\alpha = \rho\mu(\beta, \gamma)\}\\
    f(\alpha, \beta, \gamma | D) & \propto & \prod_{i = 1}^n\omega_i(t_i|\alpha, \beta, \gamma)\prod_{j = 1}^mr_j(t_j|\alpha, \beta, \gamma)
\end{eqnarray*}
onde $\omega$ e $r$ são as funcões de likelihood que representam a distribuição de \mbox{Weibull}. O logaritimo dessa função é
dada por:
\begin{eqnarray}
    \omega l_i & = & \log(\beta) + (\beta - 1)\log(t_i + \alpha) - \beta\log(\gamma) - ((t_i + \alpha)/\gamma)^{\beta} + (\alpha/\gamma)^{\beta}\\
    rl_j      & = & -((t_j + \alpha)/\gamma)^{\beta} + (\alpha/\gamma)^{\beta}
\end{eqnarray}
de forma que a função $f$ passa a ser dada por $fl$, como se segue:
\begin{eqnarray}
    fl & = & \sum^n_{i =1}\omega l_i + \sum^m_{j = 0} rl_j
\end{eqnarray}

Por outro lado, assumiu-se para este modelo que a hipotese zero pode ser dada por:
\begin{equation}
    h(\alpha, \beta, \gamma) = \rho\gamma\Gamma(1 + 1/\beta) - \alpha = 0
\end{equation}

\section{Otimização}
A etapa de otimização foi feita maximizando $fl$ sobre $\Theta_0$ sujeita a hipotese zero.
Essa etapa foi implementada utilizando a função \textrm{fmincon()} do $matlab$. Onde
o $\theta$ que maximiza $fl$ é denotado por $\theta^*$,onde $\phi = fl(\theta^*)$.
Como esta função minimiza uma função dada, foi necessário minimizar $-fl$.
A seguir é apresentada a implementação da hipotese zero:
{\linespread{1.15}\lstinputlisting[language=octave, label=sqlselect, caption={Implementação da hipotese zero}]{con.m}}

\section{Integração}
A etapa de integração foi feita utilizando-se o metodo de Monte Carlo em $Hit or Miss$. Porém, para isso,
foi necessário realizar certas mudanças de variáveis de forma a tornar os limites das integrais limitados. Pois
a etapa de integração é dada por:
\begin{equation}
    k^* = \int_0^{\infty}\int_3^{4}\int_0^{\infty}f_{\phi}(\alpha,\beta,\gamma)\mathrm{d}\alpha\mathrm{d}\beta\mathrm{d}\gamma
\end{equation}
com as mudanças de variáveis a etapa pode ser reformulada da seguinte forma:
\begin{equation}
    k^* = \int_0^{\pi/2}\int_3^{4}\int_0^{\pi/2}f_{\phi}(\tan(a),b,\tan(c))(\sec(a)\sec(c))^2\mathrm{d}a\mathrm{d}b\mathrm{d}c
\end{equation}
onde a $f_{\phi}$ é dada por:
\[
    f_{\phi} = \left\{\begin{array}{l}
                    0,\quad f(\theta) < \phi\\
                    f(\theta),\quad \mathrm{c.c.}
               \end{array}\right.
\]

A implementação de $fl$ é apresentada a seguir:
{\linespread{1.15}\lstinputlisting[language=octave, label=sqlselect, caption={Implementação da hipotese zero}]{fun.m}}

\section{Implementação}
Segue, a implementação total do programa, a qual faz uso das duas funções apresentadas nas seções anteriores
{\linespread{1.15}\lstinputlisting[language=octave, label=sqlselect, caption={Implementação da hipotese zero}]{ep4.m}}

\subsection{Resultados}
A tabela a seguir apresenta as valores de evidencia e de $\theta^*$, ou seja, $\alpha, \beta$ e $\gamma$ para dez valores
de $\rho$ distintos com $t_i$'s e $t_j$'s pré-definidos.
{\linespread{1}
    \begin{center}
        \begin{tabular}{c c c c c}
            \toprule[0.11em]
            \n{$\rho$} & \n{$ev$} & \n{$\alpha$} & \n{$\beta$} & \n{$\gamma$}\\
            \toprule[0.11em]
            0.05 & 0.9756 & 0.1116  &  3.0000  &  2.5004\\ 
            \midrule
            0.01 & 0.9603 & 0.0215  &  3.0000  &  2.4103\\
            \midrule
            0.2 & 0.9955 & 0.5112  &  3.0000  &  2.8624\\
            \midrule
            0.3 & 1 & 0.8427  &  3.0000  &  3.1458\\
            \midrule
            0.4 & 1 & 1.2781  &  3.3022  &  3.5621\\
            \midrule
            0.5 & 1 & 1.8780  &  3.8175  &  4.1547\\
            \midrule
            0.6 & 0.9986 & 2.6107  &  4.0000  &  4.8005\\
            \midrule
            0.7 & 0.9960 & 3.5135  &  4.0000  &  5.5376\\
            \midrule
            0.8 & 0.9911 & 4.6888  &  4.0000  &  6.4662\\
            \midrule
            0.9 & 0.9813 & 6.2326  &  4.0000  &  7.6402\\
            \toprule[0.11em]
        \end{tabular}
\end{center}}

\end{document}
